\section{Conclusiones}

El desarrollo de la Base de Datos que modela el problema planteado implicó un proceso iterativo: desde el modelo inicial del cual partimos, realizamos al menos cinco iteraciones hasta encontrar un modelo final. Para esto, fue necesario no sólo, criticar los modelos que fueron surgiendo, sino también consultar varias veces con el cliente de manera a entender ciertas restricciones del problema y la motivación del mismo.\\

Principalmente, lo más complicado de modelar fueron las Categorias. Pasamos por distintas ideas hasta encontrarnos con la versión final que transforma cada característica de una categoría en una entidad por sí misma. Y a su vez, todas estas se relacionan con la Modalidad. \\

Uno de los elementos claves para llegar a este modelo final fue el objetivo que nos planteamos: no sobre-cargar las restricciones del Modelo con información que pueda explayarse en el Diagrama. De este modo, el diseño es más limpio y evitamos, desde el momento de la creación de la Tabla, instanciar de manera erronea a los elementos.\\